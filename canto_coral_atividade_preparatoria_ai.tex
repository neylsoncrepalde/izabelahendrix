\documentclass[a4paper, 12pt, openright, oneside, english, brazil, article]{abntex2}
\usepackage[brazil]{babel}
\usepackage{graphicx}
\usepackage[utf8]{inputenc}
\usepackage{wrapfig}
\usepackage{lscape}
\usepackage{rotating}
\usepackage{epstopdf}
\usepackage[alf]{abntex2cite}
\usepackage[a4paper, left=3cm, right=2cm, top=3cm, bottom=2cm]{geometry}
\usepackage{indentfirst}
\usepackage{longtable}
\usepackage{amsmath}
\usepackage{verbatim}
\pagestyle{plain}

\titulo{Canto Coral - Atividade preparatória para a Avaliação Integrada}
\autor{Prof. Neylson Crepalde}
\instituicao{Curso de Música Izabela Hendrix}


\begin{document}
	
	\maketitle
	
	\section{Questão 1}
	Dentre as opções abaixo, qual melhor descreve a relação entre um coral amador e seu regente?
	
	\begin{enumerate}
		\item [A)] Uma relação voltada à performance onde o maestro é alguém que conduz os artistas que com ele trabalham.
		\item [B)] Uma relação marcada pela educação musical. Provavelmente, o maestro será o único professor que esses cantores terão.
		\item [C)] Uma relação de liderança marcada pela tomada de decisão centralizada onde há pouco espaço para opinião dos coristas.
		\item [D)] Uma relação com muitos atritos e desentendimentos na maioria das vezes.
	\end{enumerate}
	
	\section{Questão 2}
	Dentre as opções abaixo, marque aquela que descreve com maior acurácia o tipo de voz conhecida como "contralto".
	
	\begin{enumerate}
		\item [A)] Voz masculina com tessitura mais aguda e timbre mais brilhante.
		\item [B)] Voz masculina com tessitura mais grave e timbre mais escuro.
		\item [C)] Voz feminina com tessitura mais aguda e timbre mais brilhante.
		\item [D)] Voz feminina com tessitura mais grave e timbre mais escuro. 
	\end{enumerate}
	
	\section{Questão 3}
	Observe as afirmações abaixo:

	\begin{enumerate}
		\item Os arranjos corais normalmente aparecem em 4 vozes.
		\item É possível escrever arranjos em 5 vozes ou mais.
		\item É possível escrever arranjos em 3 vozes ou menos.
		\item É possível escrever arranjos apenas com vozes femininas ou masculinas.
	\end{enumerate}
	
	Quais dessas afirmações estão CORRETAS?
	
	\begin{enumerate}
		\item [A)] 1 e 3
		\item [B)] 1 e 2
		\item [C)] 1, 2 e 3.
		\item [D)] Todas estão corretas.
	\end{enumerate}
	
	\section{Questão 4}
	
	De acordo com os princípios da pirâmide de Maslow, é CORRETO afirmar que
	
	\begin{enumerate}
		\item [A)] Ações de cunho social vinculadas ao resultado sonoro não tem efeito visto que variáveis contextuais não interferem no fazer artístico de um indivíduo.
		\item [B)] Ações educacionais são mais eficientes quando acompanhadas por elementos audiovisuais.
		\item [C)] Visando um melhor aproveitamento do trabalho artístico, a atuação de um profissional da área da psicopedagogia é benéfico.
		\item [D)] O maestro se beneficiaria de uma postura mais autoritária e firme visando construir um ambiente de cooperação e obediência junto aos coristas.
	\end{enumerate}

	\section{Questão 5}
	De acordo com a experiência descrita por Rita Fucci Amato (2007), quando regia um coral formado por funcionário de diversos setores de uma indústria da cidade de São Paulo, ``foi possível primeiramente verificar uma quebra nos níveis hierárquicos estabelecidos	pelo trabalho dentro da empresa; para participar do coral só era necessário querer cantar. O gosto pelo canto estabeleceu as condições para tal quebra e criou a possibilidade de diferentes pessoas de diferentes categorias profissionais se integrarem para realizar um mesmo trabalho''.
	
	A partir da observação da autora, NÃO podemos afirmar que:
	
	\begin{enumerate}
		\item [A)] A prática de canto coral proporciona um ambiente que fomenta a igualdade, o respeito mútuo e a habilidade de lidar com a alteridade (o diferente, o outro).
		\item [B)] É possível utilizar a prática de canto coral relacionada a diversos âmbitos da vida social, como por exemplo, a igreja, o trabalho, a escola, etc.
		\item [C)] A experiência do canto coral está intrinsecamente ligada a uma visão egocentrada do indivíduo de modo que suas relações com o outro são pautadas na busca pela satisfação pessoal dissociada do senso coletivo.
		\item [D)] O regente pode ser um articulador das relações internas existentes num grupo coral.
	\end{enumerate}

	\section{Questão 6}
	
	Como se chamava a pioneira prática de educação musical utilizando vozes nas escolas no Brasil e quem foi seu principal idealizador?
	
	\begin{enumerate}
		\item [A)] Canto coletivo - Mário de Andrade
		\item [B)] Canto orfeônico - Villa-Lobos
		\item [C)] Canto coral - Osvaldo Lacerda
		\item [D)] Canto militar - Jair Bolsonaro
	\end{enumerate}
	
	
	
	\section{Questão 7}
	 Com relação à prática de canto coral, é CORRETO afirmar que:
	 
	 \begin{enumerate}
	 	\item [A)] Suas origens não são possíveis de rastrear. A prática é encontrada em todos os povos desde os tempos mais remotos.
	 	\item [B)] Suas origens remontam ao período clássico grego sendo idealizada em sua essência de prática coletiva por Pitágoras.
	 	\item [C)] Suas origens remontam ao período renascentista sendo idealizada em sua essência de prática artística por Claudio Monteverdi.
	 	\item [D)] Suas origens remontam ao período romântico sendo idealizada em sua essência por Richard Wagner.
	 \end{enumerate}
	



	
\end{document}
